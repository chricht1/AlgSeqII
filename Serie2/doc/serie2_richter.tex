\documentclass[ngerman]{mluexercise}
\usepackage[utf8]{inputenc}
\usepackage{amsfonts}
\usepackage{amsmath}
\usepackage{amssymb}
\usepackage{amsthm}
\usepackage{tikz}
\usepackage{textcomp}
\usepackage{pgfplots}
\usepackage{graphicx}
\usepackage{scalerel,amssymb}
\graphicspath{ {./images/} }
%\addbibresource{quellen.bib}
\lecture{Algorithmen auf Sequenzen II}
\studentname{Christian Mathias Richter} % Full name
\studentsymbol{amrms} % University IT shorthand symbol ("5-Steller")
\exercise{2}

\begin{document}
\maketitle
\setcounter{section}{0}
\section{}
\subsection{}
\begin{itemize}
    \item Organismus: Saccharomyces cerevisiae (Bäckerhefe) Wildtyp (rein aus der Evolution entstanden) aus der Gattung der Saccharomyces
    \item Sequenzierungsplattform: Illumina HiSeq 2000
    \item Art der Reads: 
    \begin{itemize}
        \item Single-End (nur von einem Ende der RNA ausgehende Sequenzierung), strangspezifisch (Reads können zu dem Strang von dem sie abgelesen wurden, zurückgemappt werden)
        \item RNA der gesamten Zelle ohne rRNA (rRNA-depleted)
    \end{itemize}
    \item durchschnittliche Länge der Reads: $\frac{\text{\#Bases}}{\text{\#Spots=\#Reads}} = \frac{1,4 \cdot 10^9}{13.803.030} = 101,43$ bp, festgelegte Länge bei Illumina HiSeq 2000 ist 101 bp
    \item Zustand: Steady-State, Random PCR Vervielfältigung
\end{itemize}

\subsection*{c)}

Ausschnitt vom Anfang der FastQ-Datei im Textviewer:\\

\texttt{@SRR1258470.1 HWI-ST279:269:C14WRACXX:3:1101:1413:2174 length=101\\
TCCAGAACGAAATCAACGCCGCTAATCCGTCCGACTTTCTTCAGTTCTCCGCCAACTATTTCAATAAAAGGCTGGAACAA
CAGAGAGCGTTCCTCAAGGCC\\
+SRR1258470.1 HWI-ST279:269:C14WRACXX:3:1101:1413:2174 length=101\\
@??DFFFFHAFDHFHI>FEIIEHH0DF<@GB?FGAAHE9CHAAGHH)=7)-4?8>=>;@DE;;AACCCCA@?CBB?ACBC
<<8ACC08<08<<@@3:>38<\\
@SRR1258470.2 HWI-ST279:269:C14WRACXX:3:1101:1398:2200 length=101\\
TGCCCACGTCAAGGCCGGTAAGGGTTTGATTAAGGTCAACGGTTCTCCAATCACTTTGGTTGAACCAGAAATCCTAAGAT
TCAAGGTCTACGAACCATTAT\\
+SRR1258470.2 HWI-ST279:269:C14WRACXX:3:1101:1398:2200 length=101\\
@C@FFFFDHFHHHIJGIHFHDHHIFGIG>FHDFGHIIEHFGIBF@FGDC>DGEDEEHFEHFFCE@AEDDACCCDDCDDCC
DCCDCCCCCC>@5?BDDDDDE\\
@SRR1258470.3 HWI-ST279:269:C14WRACXX:3:1101:1733:2240 length=101\\
CCAGGTATGTAAGTAGAGAATATGAAGGTGAATTAGATAATTAAAGGGAAGGAACTCGGCAAAGATAGCTCATAAGTTAG
TCAATAAAGAGTAATAAGAAC\\
+SRR1258470.3 HWI-ST279:269:C14WRACXX:3:1101:1733:2240 length=101\\
BBBFFDDEHHHHHGHIHHIIIIJJJIJJFGHHIIJJIJJJIJIJJJJJFHJJJJJJIJJJJJJJJJJJIJJJHHHHHEHF
FFFFFFEEEEECDDEFDDDDD
}\\

Man sieht zuerst eine Informationszeile zum Read, inhaltlich getrennte Informationen sind hier durch Leerzeichen getrennt. Diese Zeile beginnt mit der Read-Id beginnend mit \enquote{@}, diese enthält einen Code, die letzte Zahl nach dem Punkt scheint die Nummer des Reads, beginnend mit 1 zu sein. Dann folgt in derselben Zeile noch eine Beschreibung des Reads mit wahrscheinlich der ID des RNA-Seq-Instruments, einer Run-ID, Informationen zur Flow-Cell und die Koordinaten des Clusters wo der Read gelesen wurde. Schließlich steht noch die Länge des Reads dort, diese ist stets 101 bp.

In einer neuen Zeile ist nun die Basensequenz des Reads angegeben.
Die nächste Zeile ist eine Kopie der Informationszeile, allerdings diesmal eingeleitet mit einem \enquote{+}. Diese Zeile leitet wie in der VL beschrieben die Zeile mit den Phred-Scores ein. Die Phred-Scores als ASCII-Zeichen sind für jede einzelne Base angegeben und stets gleich \enquote{?}. Im FastQ-Report zu diesem Experiment steht, dass das Encoding im Format \enquote{Sanger / Illumina 1.9} ist. Diese verwenden die Phred+33 Codierung. \enquote{?} steht für die Zahl 63, also ist der Score stets 30. Da die Beziehung zwischen Phred-Score und Fehlerwahrscheinlichkeit $q = -10\log_{10}p$ ist. Errechnet sich Fehlerwahrscheinlichkeit errechnet sich dann zu $q=10^{-\frac{q}{10}}$. Damit ist $q=0,001$.

Nun wiederholt dieser vierzeilige Rhytmus für die weiteren Reads.

Die Informationen folgen dem in der Vorlesung beschriebenen FastQ-Format, allerdings sind neben der Read-ID noch weitere Informationen in der Zeile angegeben.


\subsection*{d)}
\begin{figure}[H]
    \centering
    \includegraphics[width=0.3\linewidth]{criteria.png}
    \caption{FastQC Ausgaben für die Kriterien}
    \label{fig:criteria}
\end{figure}
Die FastQC Ausgaben für die Kriterien sind in Abbildung \ref{fig:criteria} zu sehen.

\textbf{Per base sequence quality:}\\
Bewertung: normal\\
Man sieht, dass der Score für alle Positionen der Basen der Reads im Mittel 30 ist, und diese nicht variiert. Der Median oder das untere Quartil kann also nicht kleiner 25 sein, deshalb sollte Wertung auch normal sein.
Allerdings ist es ungewöhnlich, dass der Score immer konstant ist, gerade auch bei solch großen Readlängen. Wir sehen, dass die Sequenzierung also immer mit konstanter Qualität verlaufen ist.
% Wieso kann Readlänge immer konstant sein RNA's haben doch untersch. Länge

\textbf{Per tile sequence quality:}\\
Bewertung: normal\\
Da es keine Variation der Scores jeweils pro Position auf den Reads gab, ist es auch nicht verwunderlich, dass diese für die Flow-Cell Abschnitte bzw. Tiles ebenfalls konstant (alle schwarz) sind. Eine Warnung würde ausgegeben werden, wenn ein Flow-Cell-Abschnitt in einer Basen-Position einen Score von 2 weniger als das Mittel an der Basen-Position über alle Abschnitte hat.


\textbf{Per sequence quality scores:}\\
Bewertung: normal\\
Wegen des Ergebnisses bei der \enquote{Per base sequence quality} muss auch die Verteilung der mittleren Scores der Reads nur einen Peak bei 30 haben, was auch der Fall ist. Der Mittelwert der Verteilung ist auch >27, was einen Score von normal bedeutet.


\textbf{Per base sequence content:}\\
Bewertung: leicht abnormal\\
Die Abbildung zeigt die Anzahlen der einzelnen Basen über die Positionen in den Reads. Die Bewertung ist leicht abnormal, da FastQC Abweichungen zwischen A zu T oder G zu C an mind. einer Position von > 10 \% verzeichnet. Zwischen A und T sind die Abweichungen an Position 1 und 4 mindestens 10 \% und zwischen G und C an den Positionen 7-10 ebenfalls. 
% wieso nur Abweichungen zwischen A u. T bzw. G, C?
Ab Position 12 variieren die Anteile von A,T,G,C nicht mehr so stark wie vor dieser Position. Die Paare A, T und G, C sind hier zueinander ähnlicher, obwohl die Linien deutlich nicht aufeinander liegen. Dass ab Pos. 12 G und C deutlich seltener, als die anderen Basen vertreten sind, erklärt, wieso der G u. C Anteil in den Basis Statistics nur 40\% ist. Die Basenanteile weisen also einen relativ starken Bias auf.

Die Zahlen der Basen weichen also an fast allen Positionen deutlich von erwarteten Werten, wenn die Sequenzen zufällig wären, ab. 


\textbf{Per sequence GC content:}\\
Bewertung: leicht abnormal\\
Die Verteilung der GC-Anteile sieht zwar wie eine Normalverteilung aus, weicht aber deutlich von der erwarteten Normalverteilung ab. Die Varianz ist deutlich geringer als erwartet, daher sind die prozentualen Abweichungen der Kurven größer als 15\%, damit wird die Bewertung leicht abnormal ausgegeben.
% Wie wird erwartete Var


\textbf{Per base N content:}\\
Bewertung: normal\\
Die Anteile von nicht genau bestimmbaren Basen in den Reads an den jeweiligen Positionen ist durchweg 0\%. Dies ist ein Ergebnis einer (in dieser Bedingung) perfekten Sequenzierung und wird daher mit normal bewertet. Dies weist erneut auf keine technischen Probleme bei der Sequenzierung hin.


\textbf{Sequence Length Distribution:}\\
Bewertung: normal\\
Die Verteilung der Sequenzlängen besitzt nur einen Peak und ist daher optimal, wenn alle RNA-Fragmente mindestens die Länge von 101 haben. Dies ist bei Transkriptomanalysen allerdings oft der Fall\footnote{\url{https://knowledge.illumina.com/library-preparation/rna-library-prep/library-preparation-rna-library-prep-reference_material-list/000001243}}.


\textbf{Sequence Duplication Levels:}\\
Bewertung: sehr ungewöhnlich\\
Man sieht im zugehörigen Diagramm die Anteile von Sequenzen bzw. Reads je nach Duplikationszahl. Es wird zudem noch der Anteil über deduplizierten Sequenzen angegeben (alle Duplikate wurden entfernt). Im Titel wird auch der Anteil der deduplizierten Sequenzen in Relation zu allen Sequenzen angegeben, welcher die Bewertung ausmacht. Da dieser nur bei ca. 23\% unter 50\% liegt, wird das Ergebnis als sehr ungewöhnlich bewertet. Bei zufälligen Sequenzen gerade bei einer Länge von 101 erwarten wir nur einen sehr geringen Anteil an Duplikaten. 
% Warum sind die Kurven übereinander?
Betrachten wir die Kurven sehen wir, dass verhältnismäßig viele Sequenzen 10- bis 5000-fach im Vergleich zu den Einzigartigen dupliziert sind. Auch die Anteile an Sequenzen noch höherer Duplikation sind hoch, wenn auch geringer. Die häufig duplizierten Sequenzen gehören im Transkriptom zu aktiven bis sehr aktiven Genen und sind durchaus zu erwarten. Hier sollte also die Bewertung von FastQC im Kontext des Experiments bezüglich Fehler im Experiment also nicht unbedingt so ernst genommen werden.
% Wie kann es sein, dass Sequenzen sich widerholen, obwohl jede Seq i. d. R. Index hat? Dieser wird wahrsch. rausgefiltert?


\textbf{Overrepresented sequences:}\\
Bewertung: leicht abnormal\\
Wir bemerken, dass es einige überrepräsentierte Sequenzen gibt, angefangen bei einem Anteil von 0,3\%, viele sind nur leicht über der 0,1 \% Grenze, um gelistet zu werden. Keine Sequenz weist auf eine bekannte Kontamination hin, sind also wahrscheinlich hinsichtlich des Transkriptoms biologisch relevant. Die Bewertung lautet leicht abnormal, da Sequenzen mindestens einen Anteil an allen Reads von 0,1 \% ausmachen. Diese kann als Warnung angesehen werden und weist hier auf Vorsicht bzw. Erwägung von Kontamination hin, dies kann aber z. B. durch Zuordnung einer biologischen Funktion zur Sequenz ausgeschlossen werden. 


\textbf{Adapter Content:}\\
Bewertung: leicht abnormal\\
Es werden verschiedene Adapter für die RNA-Sequenzierung gebraucht, außerdem gibt es RNA-Adapter mit biologischer Funktion. Diese werden von FastQC detektiert und im Report mit ihrer Auftretenswahrscheinlichkeit an der jeweiligen Position angegeben.
Man sieht, dass die Adapterauftretenswahrscheinlichkeiten sich zum Ende erhöhen, aber nur der Illumina Universal Adapter tritt relativ häufig im Vergleich zu den anderen auf. Die Adapter sind für die reinen RNA-Seq Daten Störfaktoren, da der Universaladapter eine Auftrittswahrscheinlichkeit von >5\% hat, wird hier eine Warnung ausgegeben. 

%Allgemein kann man sagen, dass ein Score von 30 für alle Basen gut ist und auch Sicherheit über die Ergebnisse gibt, aber durchaus besser sein könnte. 



\subsection*{e)}
\textbf{Per base sequence content:}\\
Die großen Abweichungen der Anteile der Basen A,T bzw. G,C am Anfang der Reads könnten von dem zufälligen k-mer Priming der RNA kommen (Quelle: FastQC-Dokumentation). Das Priming ist notwendig, um im Zuge der Library Preparation aus der RNA cDNA zu machen. Genauer muss eine zufällige Basensequenz an die RNA angehängt werden, um dann daran einen Primer für das Enzym Reverse Transkriptase zu binden. Da das k-mer Priming zufällig ist, sollten eigentlich die Basenanteile relativ gleichverteilt sein, durch die PCR-Amplifizierung allerdings, welche nicht alle RNAs gleichhäufig amplifiziert, könnten die großen Abweichungen der Anteile am Anfang entstanden sein. 

Weiß man die Länge der k-mere (falls diese dieselbe Länge haben) so könnte man die ersten k Positionen der Reads ignorieren und so das Problem beheben. % Wäre dass nicht eine Art Adapter-Trimming?
Eine weitere Möglichkeit zur Behebung oder Reduktion des Problems könnte eine Umgehung oder Verringerung der PCR-Amplifizierung sein, dazu müsste mehr RNA als Material zur Sequenzierung aufgebracht werden\footnote{\url{https://www.thermofisher.com/de/de/home/life-science/cloning/cloning-learning-center/invitrogen-school-of-molecular-biology/next-generation-sequencing/dna-sequencing-preparation-illumina.html}}, dies sollte bei dem untersuchten Organismus kein großes Problem sein. 


\textbf{Per sequence GC content:}\\
Ein möglicher Grund für zu geringe Varianz der Verteilung des GC-Anteils könnten Überrepräsentierte Sequenzen sein. Diese könnten natürlicherweise im Transkriptom auftreten (also stark exprimierte Gene sein) oder durch eine zu starke PCR-Amplifizierung entstanden sein. Weniger PCR-Zyklen könnten dann wieder das Problem reduzieren.



\textbf{Sequence Duplication Levels:}\\
Da eine gewisse Menge an Sequenzduplikaten bei RNA-Seq eines Transkriptoms zu erwarten ist, da diese Sequenzen zu stark exprimierten Genen gehören könnten. Daher könnte hier keine Korrektur des Fehlers notwendig sein. 

Allerdings könnte auch wieder die PCR-Vervielfältigung die Duplikationszahl der bereits häufig vorhandenen Sequenzen noch deutlich erhöht haben und damit die eigentlichen Verhältnisse der RNAs in der ursprünglichen Probe verzerrt haben. Daher könnten die Duplikationszahlen der Sequenzen mit sehr hoher Zahl verringert werden, wenn weniger PCR-Zyklen durchlaufen werden.

\textbf{Overrepresented sequences:}\\
Da die überrepräsentierten Sequenzen nicht als Verunreinigung identifiziert wurden, könnte es sich also um RNAs sehr stark exprimierter Gene handeln, diese hohen Anteile repräsentieren die hohe Duplikationszahl, daher ist die Lösungsmöglichkeit analog zu Sequence Duplication Levels.



\textbf{Adapter Content:}\\
Wenn die RNA-Sequenzen kürzer als die Readlänge sind oder wenn diese nach der Zerlegung in der Library Preparation kürzer sind, so werden Adapter am 3' Ende der Reads mitsequenziert\footnote{\url{https://www.ecseq.com/support/ngs/trimming-adapter-sequences-is-it-necessary}}. Dies könnte hier mit dem Illumina Universal Adapter der Fall sein. 
Eine Lösung wäre ein Adapter Trimming, was die Adaptersequenz in den Reads entfernt.

%\printbibliography

%FASTQC REPORT MIT ABGEBGEN?!
\end{document}