\documentclass[ngerman]{mluexercise}
\usepackage[utf8]{inputenc}
\usepackage{amsfonts}
\usepackage{amsmath}
\usepackage{amssymb}
\usepackage{amsthm}
\usepackage{tikz}
\usepackage{textcomp}
\usepackage{pgfplots}
\usepackage{graphicx}
\usepackage{scalerel,amssymb}
\graphicspath{ {./images/} }
%\addbibresource{quellen.bib}
\lecture{Algorithmen auf Sequenzen II}
\studentname{Christian Mathias Richter} % Full name
\studentsymbol{amrms} % University IT shorthand symbol ("5-Steller")
\exercise{4}

\begin{document}
\maketitle
\setcounter{section}{0}
\section{}
\section{}
\subsection{}
Trimmomatic Aufruf:
\texttt{input=../../Data/SRR1258470/SRR1258470.fastq.gz\\
output=SRR1258470_trimmed.fastq.gz\\
trimmomatic SE -phred33 -trimlog SRR1248470_trimlog.log $input $output HEADCROP:10 TAILCROP:10 LEADING:28 TRAILING:28 MINLEN:20}

\subsection{}

\textbf{Allgemeine Daten:}\\
Die Sequenzlängen haben sich von 101 auf 20-81 geändert, der GC-Gehalt ist nur um 1\% auf 39\% gesunken. 
Es wurden 13803030-13800105=2925 Reads entfernt, also ein sehr kleiner Anteil.


\textbf{Per base sequence quality:}\\
Bewertung: normal\\
Da an beiden Enden jeweils mindestens 10 bp der Reads entfernt wurden sind die Scores dieser Positionen, 
welche i. A. etwas niedriger waren, als die mittleren, nun nicht mehr zusehen. 
Außerdem wurden von beiden Enden Basen mit Qualität <28 entfernt, 
dadurch sind die Scores an den Rändern der getrimmten Reads sind etwas höher,
man kann mit bloßem Auge aber kaum einen Unterschied in den Boxplots der verbliebenen Positionen Qualitätsscores erkennen.
Die Bewertung war vor der Trimming schon gut und kann sich nicht verlechtert haben. 


\textbf{Per tile sequence quality:}\\
Bewertung: leicht abnormal\\
Es ist kein großer Unterschied in den verbliebenen Positionen zu erkennen. 
Nur manche Tiles mit besserer Qualität als der Durchschnitt heben sich noch etwas mehr vom Durchschnitt ab.  
Für diese Tiles, werden teilweise aber nur Positionen mit durschnittlicher Qualität entfernt, 
wordurch die Ursache für die Verbesserung zunächst nicht klar wird.
Die Bewertung hat sich allerdings verbessert, sodass die Stärke der roten Farbe doch abgenommen haben sollte.



\textbf{Per sequence quality scores:}\\
Bewertung: normal\\
Der Peak der Scoreverteilung ist nun bei 39 statt bei 38, dies ist ungewöhnlich, 
da eigentlich durch das Cropping am Anfang und Ende nach der \enquote{Per base sequence quality}-Abbildung 
nur wenige Basen mit einem Score von 38 entfernt wurden.
Der mittlere Quaitätsscore hat sich also verbessert, was nicht verwunderlich ist.
Außerdem ist der kleinste Score nun bei 18 statt 2, da wir die Basen am Rand mit den niedrigsten Scores entfernt haben.


\textbf{Per base sequence content:}\\
Bewertung: normal\\
Durch das Cropping am Anfang wurde ein Großteil der Positionen am Anfang mit den großen Abweichungen 
der Basengehalte zu den restlichen Positionen entfernt.
Durch das weitere Trimming von Basen am Anfang mit niedriger Qualität konnten auch die AT- und vorallem die GC-Gehalte
direkt nach dem Cropping angeglichen werden.

Durch das Entfernen der letzten 10 Positionen ist auch das Auseinanderlaufen der AT-Gehalte am Ende nicht mehr so stark sichtbar.

Ansonsten sehen die Kurven ähnlich aus, in mittleren Positionen sollte sich auch nicht viel verändert haben.


\textbf{Per sequence GC content:}\\
Bewertung: leicht normal\\
Die Plots zwischen ungetrimmten und getrimmten Reads sehen relativ ähnlich zueinander aus, obwohl bei den getrimmten Reads, 
ist die empirische Verteilung zur theoretischen etwas mehr nach links verschoben und weist keinen so starken Peak mehr auf.
Es wird auch sichtbar, dass beide Verteilungen bei den getrimmten Reads etwas nach links gerückt sind.
Dies ist erklärbar dadurch, 
dass die Teile der Reads am Anfang mit durchschnittlich deutlich höherem GC-Gehalt, als an den späteren Positionen entfernt wurden.
Am Anfang war zusätzlich auch der mittlere T-Gehalt stark gesenkt.
Die abstumpfung des Peaks könnte daher kommen, 
dass durch das Trimming mit dem Schwellwert für den Score die Sequenzen jetzt unterschiedliche Längen haben.
Dadurch könnte auch mehr Varianz bei den GC-Gehalten entstehen.



\textbf{Per sequence GC content:}\\
Bewertung: normal\\
Es waren vorher auch schon so gut wie gar keine Basen nicht-identifizierbar, man sieht keine Verbesserung.


\textbf{Sequence Length Distribution:}\\
Bewertung: leicht abnormal\\
Man sieht an der Längenverteilung der getrimmten Reads, dass fast nur Reads der Länge 80-81 vorkommen.
Betrachtet man aber die Qualitäten, so weiß man, dass diese Reads fast nur die Länge 81 haben müssen, 
da die Reads vor dem Trimmen alle Länge 101 hatten
und zusätzlich zu dem Cropping kaum Reads weiterhin getrimmt worden sein dürften wegen der höheren Qualität.  
Komischerweise geht die x-Achse der Verteilung bis zur Länge 18-19, sie sollte nur bis 20 gehen.
Die Warnung, die Fastqc hier ausgibt kann ignoriert werden, da es gerade gewollt war, 
dass Sequenzen getrimmt werden.


\textbf{Sequence Duplication Levels:}\\
Bewertung: stark abnormal\\
Die Duplikatzahlen sind nach der Behandlung kaum unterschiedlich, 
lediglich bei der Häufigkeit der einfach vorhandenen Sequenzen sieht man eine deutliche Reduktion. 
Da nur aber 0,02\% der Sequenzen entfernt wurden, bedeutet dies, 
dass durch das Trimming mehr Duplikate entstanden sein müssen, was man auch in der leichten Erhöhung der Duplikationslevel
bei 9, >10, und >50 auch sehen kann.




\end{document}